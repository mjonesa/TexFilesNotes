\documentclass{article}


\usepackage[margin=1in]{geometry}
\begin{document}

\title{Intro to Heat of Formation}
\date{}
\author{Martin Jones}
\maketitle
\thispagestyle{empty}

The Heat of Formation is conveniently named because it represents the amount of energy to put together or `form' a chemical species. It is applicable to chemical reactions especially and is quantified as the change of enthalpy for such a reaction. One is able to calculate it numerically, or experimentally from the two respective formulae.

\begin{equation}
\Delta H^{\Theta}=\sum \Delta v_{p}\Delta H^{\Theta}_{f}(products)-\sum \Delta v_{r}\Delta H^{\Theta}_{f}(reactants)
\end{equation}
\begin{equation}
\Delta H^{\Theta}=q_{r}=mc_{p}\Delta T
\end{equation}
Where $\Delta H^{\Theta}$ is the enthalpy of formation, $v_{p,f}$ is the stoichiometric constant for products or reactants, $\Delta H^{\Theta}_{f}$ is the standard enthalpy of formation for the products or reactants, $q_{r}$ is the heat of reaction, $m$ is the mass, $c_p$ is the specific heat, and $T$ is the temperature.  


\end{document}













