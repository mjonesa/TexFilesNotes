\documentclass{article}


\usepackage[margin=1in]{geometry}
\begin{document}

\title{Thermochemical Ablation (TCA) Model}
\date{}
\author{Martin Jones}
\maketitle
\thispagestyle{empty}

We start by defining an open system in our liver that has the following energy balance:

\begin{equation}
\label{master}
\dot{E}_{in}-\dot{E}_{out}=\dot{E}_{stor}-\dot{E}_{gen}
\end{equation}

Recalling we have the following equations for the enthalpy of formation and heat of reaction, we are able to form our equation,
\begin{equation}
\Delta H^{\Theta}=\sum \Delta v_{p}\Delta H^{\Theta}_{f}(products)-\sum \Delta v_{r}\Delta H^{\Theta}_{f}(reactants)
\end{equation}
\begin{equation}
\label{heatofreaxion}
\Delta H^{\Theta}=q_{r}=mc_{p}\Delta T
\end{equation}
Where $\Delta H^{\Theta}$ is the enthalpy of formation, $v_{p,f}$ is the stoichiometric constant for products or reactants, $\Delta H^{\Theta}_{f}$ is the standard enthalpy of formation for the products or reactants, $q_{r}$ is the heat of reaction, $m$ is the mass, $c_p$ is the specific heat, and $T$ or $u$ is the temperature.  

The LHS can be related to the heat flux across the boundary of the system. It turns out to be $-\vec{\nabla}\cdot(k\vec{\nabla u})$. And there is no stored heat, so $\dot{E}_{stor}=0$. Now, employing equation (\ref{heatofreaxion}), and taking the time partial derivative we obtain a formula for the generated heat: (there is no blood perfusion because the liver is dead)???
\begin{equation}
\label{heatgen}
\dot{E}_{gen}=\dot{q}_{r}=mc_{p}u_{t}
\end{equation}
Balancing dimensions and plugging into Equation (\ref{master}) we have, 
\begin{equation}
\rho c_{p}u_{t}=\vec{\nabla}\cdot(k\vec{\nabla u})
\end{equation}
Where $\rho$ is the density of the tissue, blood, salt, and water mixture. And $c_{p}$ is the specific heat of just the blood, tissue.




\end{document}













