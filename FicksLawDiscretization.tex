\documentclass{article}


\usepackage[margin=1in]{geometry}
\begin{document}

\title{Fick's Law}
\date{}
\author{Martin Jones}
\maketitle
\thispagestyle{empty}
Diffusion is a transport phenomenon that involves mass transport or mixing that does not require bulk motion. Fick's law for transient phenomena is written thus,

\begin{equation} 
\frac{\partial \phi}{\partial t} = \vec{\nabla} \cdot (D(\vec{x})\vec{\nabla} \phi)
\end{equation} 
Where $\phi$ and $D$ are concentration and diffusivity, repectively, scalar functions that depend on space and time. And we have $\phi(\vec{x},0)=\phi_{0}(\vec{x})$ and $\vec{\nabla}\phi \cdot \vec{\mathbf{n}}=0$ on the surface for our boundary conditions.

Now multiply both sides by an arbitrary test function $V(x)$ and integrate over a volume $\Omega$. Also, integrating by parts on the RHS, the equation evolves to the following form,  (making use first of the following theorem):

\begin{displaymath}
\int_{\Omega}V\vec{\nabla}\cdot(D\vec{\nabla}\phi)d\Omega=\int_{S}V(D\vec{\nabla}\phi\cdot \vec{\mathbf{n}})dS-\int_{\Omega}D(\vec{\nabla}V\cdot \vec{\nabla} \phi)d\Omega
\end{displaymath}

 \begin{equation} 
\int_{\Omega} \frac{\partial \phi}{\partial t} V(x)d\Omega = -\int_{\Omega}D(\vec{x})\vec{\nabla}V \cdot\vec{\nabla} \phi d\Omega
\end{equation}
Now we will discretize and write with basis functions in N dimensional space (using Einstein notation) where N is on the order of the number of elements in the mesh,

\begin{equation}
\phi (\vec{x},t_{k})=\phi_{i}^{k}\psi_{i}
\end{equation}and
\begin{equation}
V(\vec{x})=V_{j}\psi_{j}
\end{equation}

Using the finite difference formula for the time derivative and plugging in we get,

 \begin{equation} 
\int_{\Omega} \left(\frac{\phi_{i}^{k+1}-\phi_{i}^{k}}{\Delta t}\right)\cdot \psi_{i} (V_{j}\cdot \psi_{j })d\Omega = -\int_{\Omega}D(\vec{x})\vec{\nabla}(V_{j}\psi_{j}) \cdot\vec{\nabla} (\phi_{i}^{k+1}\psi_{i}) d\Omega
\end{equation}
Now we may write N equations with N unknowns due to the arbitrariness of V (1 on the $i^{th}$ node and zero every where else). One gets,

\begin{equation}
\left(\frac{1}{\Delta t}\vec{M}+\vec{K}\right)\vec{\phi}^{k+1}=\frac{1}{\Delta t}\vec{M}\vec{\phi}^{k}
\end{equation}
Where M is given by,
\begin{equation}
M_{i,j}=\int_{\Omega}\psi_{i}\psi_{j}d\Omega
\end{equation}
and K is given by,
\begin{equation}
K_{i,j}=\int_{\Omega}D(\vec{x})\vec{\nabla}\psi_{i}\cdot \vec{\nabla}\psi_{j}d\Omega
\end{equation}

\end{document}













